% Vorgaben Assignment aus Studienheft SQL03/SQLD302
% Formatvorgaben fuer den Text
% Umfang: (inkl. Abbildungen und Tabellen, aber ohne Deckblatt, Gliederung und Literaturverzeichnis, Eidesstattliche Erklaerung)
%    Assignments: 8 - 12 Seiten
%    Projektberichte: ca. 25 Seiten
%    Bachelor-Arbeit: ca. 40 bis 60 Seiten (Qualität ist hier wichtiger als die "korrekte" Seitenzahl!)
%    Master-Arbeit: ca. 60 bis 80 Seiten (Qualität ist hier wichtiger als die "korrekte" Seitenzahl!)
%    Diplomarbeit: ca. 50 bis 70 Seiten (Qualität ist hier wichtiger als die "korrekte" Seitenzahl!)
% Bearbeitungsdauer:
%    Assignments: 2 Monate
%    Projektberichte: 4 Monate
% Zeilenabstand: 1,5x (in Tabellen wahlweise auch 1x)
% Schriftart: frei (Times New Roman oder Arial sind empfohlen)
% Schriftgrad: 12 pt (in Tabellen wahlweise auch 10 pt)
% Variablen, physikalische Groessen und Funktionszeichen werden kursiv gedruckt.
% Korrekturrand: links: 4,5 cm (2,5 cm für Assignments), rechts 2,0 cm, oben und unten jeweils 3,0 cm
% Deckblatt: (Adresse, E-Mail-Adresse, Immatrikulationsnummer, Modulbezeichnung, Thema, Datum, Felder für Korrektor)
% Gliederung (1 Seite)
% Literaturverzeichnis (z. B. Lehrbuecher, aktuelle Fachartikel recherchieren... Assignments sollten mit 3 - 5 Literaturquellen belegt werden)
% Eidesstattliche Erklaerung, unterschrieben und fest eingebunden (Ausnahme: bei Assignments wird diese elektronisch beim Upload abgegeben)


%
%% Document Class (Koma Script) -----------------------------------------
%% Doc: scrguien.pdf
\documentclass[%
   %draft=true,     % draft mode (no images, layout errors shown)
   draft=false,     % final mode 
%%% --- Paper Settings ---
   paper=a4,
   paper=portrait, % landscape
   pagesize=auto, % driver
%%% --- Base Font Size ---
   fontsize=12pt,%
%%% --- Koma Script Version ---
   version=last, %
%%% --- Global Package Options ---
   ngerman, % language (passed to babel and other packages)
   parskip,
   numbers=noenddot,
   listof=totoc,        % add lists of figures and tables to Table of Contents
   bibliography=totoc,  % add cited works to Table of Contents
]{scrreprt} % Classes: scrartcl, scrreprt, scrbook\usepackage[ngerman]{babel}

\newif\ifsperrvermerk
\newif\ifappendix
\newif\iflistoffigures
\newif\iflistoftables
\newif\ifacronym
\newif\iflistofformeln
\newif\ifassignment
\newif\ifabschlussarbeit
\newif\ifglossary

% Einstellungen für das Gesamtdokument
% Hinweis: Wenn diese Arbeit in einem git repository geschrieben und auf github gesichert wird, besteht die Gefahr,
%          persönliche Daten (insbesondere Adresse und Matrikelnummer) unabsichtlich zu veröffentlichen.
% Um dies zu verhindern, sollten kritische Daten in einer oder mehreren separaten Dateien namens "secrets.tex" gesetzt werden (einfach von unten dorthin kopieren, einstellungen.tex muss nicht verändert werden).
% Diese können entweder im Projektverzeichnis selbst liegen, im Verzeichnis direkt darüber, oder im HOME-Verzeichnis ($HOME oder %USERPROFILE%).
% Es kann gemischt werden (z.B. Adresse in HOME, Studiengang im Projekt selbst), der jeweils zuerst gesetzte Wert hat Vorrang:
\IfFileExists{secrets.tex}{\input{secrets.tex}}{}                   % im Projekt-Ordner selbst (leicht zu packen/sichern/transferieren des Projekts, besonders für Dinge wie Titel und Thema des assignments)
\IfFileExists{../secrets.tex}{\input{../secrets.tex}}{}             % direkt oberhalb des Projekt-Ordners (gültig für mehrere nebeneinander liegende Assignments, daher besonders für unveränderliche Daten wie Adresse)
\IfFileExists{\string~/secrets.tex}{\input{\string~/secrets.tex}}{} % aus HOME Verzeichnis (gültig für alle Assignments auf dem Rechner egal wo sie liegen)
% Im Folgenden nun die defaults für alle Werte, die nicht in einer der secrets.tex Dateien gesetzt wurden:


%%% Themenbezogene Daten (bei jedem neuen Assignment zu aktualisieren):

%Titel
\providecommand*{\Titel}{Thema des Assignments}

%PDF Beschreibung
\providecommand*{\pdfsubject}{Eine kurze Beschreibung, worum es geht}

%Betreff
\providecommand*{\Arbeitstyp}{Assignment im Modul ABC01}

%Betreuer
\providecommand*{\Betreuer}{Prof. Dr. Mustermann}

%Firma, in der die Arbeit absolviert wurde (für Sperrvermerk)
\providecommand*{\Firma}{Mustermann AG}

%Bearbeitungszeit
\providecommand*{\Bearbeitungszeit}{8 Wochen}

%PDF Keywords
\providecommand*{\pdfkeywords}{akad, assignment, meta, information, pdf, hyperref, latex}



%%% personenbezogene Daten (bleiben gleich von Assignment zu Assignment, sollten vorzugsweise in secrets.tex gesetzt werden, siehe unten):

%Vor- und Nachname
\providecommand*{\Name}{Max Mustermann}

%Straße und Hausnummer
\providecommand*{\Strasse}{Musterstr. 1a} 

%Plz und Ort
\providecommand*{\PlzOrt}{12345 Musterhausen} 

%Email 
\providecommand*{\Email}{max.mustermann@akad.de} 

%Immatrikulationsnummer
\providecommand*{\Immatrikulationsnummer}{123456}

%Studiengang
\providecommand*{\Studiengang}{IT-Management}

%Akademischer Grad
\providecommand*{\Grad}{Master of Science (M. Sc.)} 
%\providecommand*{\Grad}{Master of Engineering (M. Eng.)} 
%\providecommand*{\Grad}{Bachelor of Science (B. Sc.)} 
%\providecommand*{\Grad}{Bachelor of Engineering (B. Eng.)} 



%%% generelle Einstellungen zur Darstellungsform
% Überschrift des Literaturverzeichnisses
\providecommand*{\prefbiblioname}{Literaturverzeichnis}

%% Nicht benötigte Zeilen mit % auskommentieren oder löschen:

%% darzustellende Verzeichnisse (können nicht in secrets.tex überschrieben werden):
\listoffigurestrue  %% Abbildungsverzeichnis 
\listoftablestrue   %% Tabellenverzeichnis
\acronymtrue        %% Abkürzungsverzeichnis
\glossarytrue       %% Glossar
% \listofformelntrue  %% Formelverzeichnis
\appendixtrue       %% Anhang

%% Diese Vorlage wird für ein Assignment benutzt (statt für eine Abschlussarbeit)
\assignmenttrue

%% dieser Bericht beinhaltelt sensible Firmendaten und benötigt einen Sperrvermerk:
%\sperrvermerktrue
%% Sperrvermerke beziehen sich im Text direkt auf die Art des Berichts. Sollte ein Sperrvermerk notwendig sein, muss hier die richtige Wortwahl definiert werden:
%% Projektbericht:
%\providecommand*{\SperrvermerkBausteinEins}{Der nachfolgende Projektbericht}
%\providecommand*{\SperrvermerkBausteinZwei}{des Projektberichts}
%\providecommand*{\SperrvermerkBausteinDrei}{Der Projektbericht}
%% Diplom-/Master-/Bachelor-Arbeit:
%\providecommand*{\SperrvermerkGrad}{Diplom}
%\providecommand*{\SperrvermerkGrad}{Master}
%\providecommand*{\SperrvermerkGrad}{Bachelor}
%\providecommand*{\SperrvermerkBausteinEins}{Die nachfolgende \SperrvermerkGrad-Arbeit}
%\providecommand*{\SperrvermerkBausteinZwei}{der \SperrvermerkGrad-Arbeit}
%\providecommand*{\SperrvermerkBausteinDrei}{Die \SperrvermerkGrad-Arbeit}


% Allgemeine Präambel für die Einbindung von Paketen
% dummy comment for file-wide intellisense errors
%\usepackage{graphicx} % enables use of eps graphics (encapsulated PostScript). Activate if needed.
%\usepackage{newtx} % replacement of previously used "times" package (using Times font as default)
\usepackage{babel}
\usepackage{supertabular}
\usepackage{wrapfig}
\usepackage{multirow}
\usepackage[onehalfspacing]{setspace}
\usepackage{scrhack}  % fix float warning of KOMA produced when including listings
\usepackage{listings}
\usepackage{mathptmx}
\usepackage{geometry}
\usepackage{helvet}
\usepackage{courier}
\usepackage{setspace}
\usepackage{textcomp}
\usepackage[T1]{fontenc}
\usepackage[utf8]{inputenc}
\usepackage{float} % Notwendig fuer figure[h]
\usepackage[german=quotes]{csquotes}
\usepackage[style=alphabetic]{biblatex} % alternative for sort: iso-authoryear
\usepackage{pdfpages}
\usepackage{calc} % for calculations with text width
% Fuer Schriftart Arial
%\usepackage[scaled]{uarial}

% Installation der Arial Schriftart unter Linux.
% wget http://tug.org/fonts/getnonfreefonts/install-getnonfreefonts
% texlua install-getnonfreefonts
% getnonfreefonts -r
% getnonfreefonts arial-urw


%% Einstellungen fuer Quellcode Highlighting
\usepackage{color}
\definecolor{mygreen}{rgb}{0,0.6,0}
\definecolor{mygray}{rgb}{0.5,0.5,0.5}
\definecolor{mymauve}{rgb}{0.58,0,0.82}

\lstset{
  backgroundcolor=\color{white},   % choose the background color; you must add \usepackage{color} or \usepackage{xcolor}
  basicstyle=\footnotesize,        % the size of the fonts that are used for the code
  breakatwhitespace=false,         % sets if automatic breaks should only happen at whitespace
  breaklines=true,                 % sets automatic line breaking
  captionpos=b,                    % sets the caption-position to bottom
  commentstyle=\color{mygreen},    % comment style
  deletekeywords={...},            % if you want to delete keywords from the given language   % chktex 11 ignore suggestion to use \ldots
  escapeinside={\%*}{*)},          % if you want to add LaTeX within your code    % chktex 9 ignore unmatched parenthesis
  extendedchars=true,              % lets you use non-ASCII characters; for 8-bits encodings only, does not work with UTF-8
  % frame=single,                    % adds a frame around the code
  keepspaces=true,                 % keeps spaces in text, useful for keeping indentation of code (possibly needs columns=flexible)
  keywordstyle=\color{blue},       % keyword style
  language=Octave,                 % the language of the code
  morekeywords={*,...},            % if you want to add more keywords to the set   % chktex 11 ignore suggestion to use \ldots
  numbers=left,                    % where to put the line-numbers; possible values are (none, left, right)
  numbersep=5pt,                   % how far the line-numbers are from the code
  numberstyle=\tiny\color{mygray}, % the style that is used for the line-numbers
  rulecolor=\color{black},         % if not set, the frame-color may be changed on line-breaks within not-black text (e.g. comments (green here))
  showspaces=false,                % show spaces everywhere adding particular underscores; it overrides 'showstringspaces'
  showstringspaces=true,           % underline spaces within strings only
  showtabs=true,                   % show tabs within strings adding particular underscores
  stepnumber=1,                    % the step between two line-numbers. If it's 1, each line will be numbered
  stringstyle=\color{mymauve},     % string literal style
  tabsize=2,                       % sets default tabsize to 2 spaces
  title=\lstname,                  % show the filename of files included with \lstinputlisting; also try caption instead of title
  belowskip= 0pt,
}  % chktex 10 suppress false positive caused by a deliberately unmatched ')' above

% PDF Einstellungen für Verlinkungen

\usepackage[
  pdftitle={\Titel},
  pdfsubject={\pdfsubject},
  pdfauthor={\Name},
  pdfkeywords={\pdfkeywords}
  hyperfootnotes=false,
  colorlinks=true,
  linkcolor=black,
  urlcolor=black,
  citecolor=black
]{hyperref}

%%% Abkürzungsverzeichnis (Glossar) Neues Paket (kann nomencl und acronym ersetzen)
% muss nach hyperref eingebunden werden, um das Paket zu nutzen
% Abkürzungen werden nur im Glossar angezeigt, wenn sie im Dokument mindestens einmal genutzt wurden
\usepackage[
  % style=long,
  toc, % Glossar erscheint im Inhaltsverzeichnis
  acronym, % Setzt Akronyme in ein gesondertes Verzeichnis
  % footnote, % Setzt eine Fußnote beim ersten verwendet wird
  % nomain,
  % style=altlist,
  nopostdot, % löscht den schlusspunkt nach jeder description
]{glossaries}
\setglossarystyle{super}
\makeglossaries{} % Glossar generieren

\newfloat{Formel}{H}{for}

%% FAQ Umgebung, z.B. für Interview Protokolle im Appendix
\newenvironment{faq}{}{}
\DeclareSectionCommand[
  runin=false,                                        % start the answer in a new line
  afterskip=0.25\baselineskip plus -1ex minus -.2ex,  % chktex 1 commands cannot be terminated with curly braces in arguments
  beforeskip=-2.5ex plus -1ex minus -.2ex,
  indent=0pt,
  level=4,
  font=\usekomafont{paragraph}\itshape, %% using the same font as paragraph, but italic
  tocindent=10em,
  tocnumwidth=5em,
  counterwithin=subsubsection,
  style=section,
]{question}
\newcommand{\faqitem}[2]{\question{#1}{\setlength{\leftskip}{\parindent}#2\par}}

\renewcommand\UrlFont{\color{black}\rmfamily\itshape} % chktex 6 ignore missing '\/' after \itshape

\renewcommand{\familydefault}{\rmdefault}
\newcommand{\bflabel}[1]{\normalfont{\normalsize{#1}}\hfill}

% Sonstige Hilfsfunktionen
%% Definition for Codeschnipsel im Fließtext
\newcommand{\code}{\texttt}

%% Todos mithilfe eines Rahmens hervorheben
\newcommand{\todo}[1]{\fbox{\parbox{\textwidth-\fboxsep*6-\unitlength}{\textbf{To do:} #1}}}
%\newcommand{\todo}[1]{}

%% eine Fußnote als Platzhalter für ein Zitat
\newcommand{\citationneeded}{\footnote{\textit{missing citation}}}

%% Footnote without number
\makeatletter
\def\blfootnote{\xdef\@thefnmark{}\@footnotetext}
\makeatother

% Zitiermakros
\newcommand{\mycite}[2]{„#1“~\autocite{#2}}
\newcommand{\mycitetranslate}[2]{„#1“~\autocite[eigene Übersetzung]{#2}}
\newcommand{\mycitepage}[3]{„#1“~\autocite[#3]{#2}}
\newcommand{\mycitepagetranslate}[3]{„#1“~\autocite[S. #3 eigene Übersetzung]{#2}}
\newcommand{\myreferencepage}[2]{\autocite[#2]{#1}}
\newcommand{\vglpage}[2]{\autocite[vgl.][#2]{#1}}
\newcommand{\vglmin}[2]{\autocite[vgl.][#2]{#1}{}}
\newcommand{\vgl}[1]{\autocite[vgl.][]{#1}}
\newcommand{\vglcph}[2]{\autocite[vgl.][Abschnitt #2]{#1}}
\newcommand{\citesep}{$^, $}

% Style Einstellungen
%% Für Codeblöcke mit Syntax-Highlighting
%% http://www.ctan.org/tex-archive/macros/latex/contrib/minted/
%% Einkommentieren fuer Minted Syntax Highlighting
%\usepackage{minted}
%\definecolor{bg}{rgb}{0.95,0.95,0.95}

\makeatother

\ifassignment
    \geometry{a4paper, left=25mm, right=20mm, top=30mm, bottom=30mm, head=21.74998pt}
\else
    \geometry{a4paper, left=45mm, right=20mm, top=30mm, bottom=30mm, head=21.74998pt}
\fi
\setlength{\footheight}{21.74998pt}

\renewcommand*{\chapterheadstartvskip}{\vspace*{0\baselineskip}}% Abstand einstellen

\pagenumbering{roman}


\usepackage[automark,headsepline]{scrlayer-scrpage}

\clearpairofpagestyles
\cfoot[\pagemark]{\pagemark}
\lehead{\headmark}
\rohead{\headmark}

\pagestyle{scrheadings}

\newacronym{url}{URL}{Uniform Resource Locator}
\newacronym{css}{CSS}{Cascading Style Sheets}
\newacronym{mituni}{MIT}{Massachusetts Institute of Technology}
\newacronym[plural=Abk,firstplural=Abkürzungen (Abk)]{abk}{Abk}{Abkürzung} % this shows how to define plurals for German words. Default is to just append s, which is fine for English and for acronyms, so we keep that default.

% just like in cited sources, any acronyms or terminology not actually used in the text will not be displayed in the final PDF, so you can add any terms you think you will need..
\newglossaryentry{pi}{
    name=$\pi$,
    description={Die Kreiszahl},
    sort=Pi
}
\newglossaryentry{lorem}{
    name=Lorem Ipsum,
    description={Lorem ipsum dolor sit amet, consetetur sadipscing elitr, sed diam nonumy eirmod tempor invidunt ut labore et dolore magna aliquyam erat, sed diam voluptua.},
    sort=Lorem Ipsum
}


\addbibresource{literatur.bib}

% BEGIN DOCUMENT %%%%%%%%%%%%%%%%%%%%%%%%%%%%%%%%%%%%%%%%%%%%%%%%%%%%%%%%%%%%%%%
\begin{document}
%%%%%%%%%%%%%%%%%%%%%%%%%%%%%%%%%%%%%%%%%%%%%%%%%%%%%%%%%%%%%%%%%%%%%%%%%%%%%%%%

% TITEL PAGE %%%%%%%%%%%%%%%%%%%%%%%%%%%%%%%%%%%%%%%%%%%%%%%%%%%%%%%%%%%%%%%%%%%

\begin{titlepage}
  \newgeometry{left=25mm, right=20mm, top=35mm, bottom=30mm}
  \begin{center}
    \thispagestyle{empty}

    \Large{\textbf{\Titel}}
    \vfill
    \onehalfspacing{}

    \large{\Arbeitstyp}

    \vfill
    \normalsize

    von

    \vspace{.5cm}
    \large{\Name}
    \normalsize
    \vfill

    \ifassignment       % chktex 1 don't terminate \if-commands with {} as it breaks auto-indentation
    \else
      Zur Erlangung des akademischen Grades \\
      \textbf{\Grad}
      \vfill
    \fi

    Im Studiengang \Studiengang{} \\
    an der staatlich anerkannten AKAD Hochschule Stuttgart

    \vfill
    \vfill

    \today

    \vfill

    \includegraphics[scale=0.35]{resources/akad_logo.png}

  \end{center}

  \vfill
  \begin{spacing}{1.2}
    \begin{tabbing}
      \hspace{9cm}     \= \kill
      \textbf{Bearbeitungszeit}       \>  \Bearbeitungszeit{} \\
      \textbf{Betreuer}               \>  \Betreuer{} \\
      \textbf{Immatrikulationsnummer} \>  \Immatrikulationsnummer{} \\
      \textbf{E-Mail}                 \> \href{mailto:\Email}{\Email} \\
      \textbf{Adresse}                \> \Strasse{} \\
      \space                          \> \PlzOrt{}
    \end{tabbing}
  \end{spacing}
  \restoregeometry{}
\end{titlepage}

%%%%%%%%%%%%%%%%%%%%%%%%%%%%%%%%%%%%%%%%%%%%%%%%%%%%%%%%%%%%%%%%%%%%%%%%%%%%%%%%

\normalsize

\ifsperrvermerk
   % this work contains restricted information (usually company-internal information)
   \begin{spacing}{1.5} % Zeilenabstand: 1,5 fuer den Sperrvermerk
      \begin{center}
         {\Large Sperrvermerk}
         \vspace*{4cm}
      \end{center}
      \noindent
      \SperrvermerkBausteinEins{} enthält vertrauliche Daten der \Firma. Veröffentlichungen oder Vervielfältigungen \SperrvermerkBausteinZwei{} -- auch nur ansatzweise -- sind ohne ausdrückliche Genehmigung der Geschäftsleitung der \Firma{} nicht gestattet. \SperrvermerkBausteinDrei{} ist nur den Gutachtern sowie den Mitgliedern des Prüfungsausschusses zugänglich zu machen.
   \end{spacing}
   \clearpage
\fi

\begin{spacing}{1.0} % Verzeichnisse werden mit einzeiligem Abstand gesetzt

% Inhaltsverzeichnis %%%%%%%%%%%%%%%%%%%%%%
\tableofcontents

% Abbildungsverzeichnis %%%%%%%%%%%%%%%%%%%%%%
\iflistoffigures
\listoffigures 
\fi

% Tabellenverzeichnis %%%%%%%%%%%%%%%%%%%%%%
\iflistoftables
\listoftables
\fi

% Abkürzungsverzeichnis %%%%%%%%%%%%%%%%%%%%%%
\ifacronym
\printglossary[type=\acronymtype,title=Abkürzungsverzeichnis]
\fi

\ifglossary
\printglossary[title=Glossar]
\fi

% Formelverzeichnis %%%%%%%%%%%%%%%%%%%%%%
\iflistofformeln
\listof{Formel}{Formelübersicht}
\newpage
\fi


\end{spacing} 

\clearpage

\newcounter{romanPagenumber} 
\setcounter{romanPagenumber}{\value{page}} % Roemische Seitenanzahl speichern.

% \nocite{*} % enable this if you want to list all your literature, even if it wasn't actually cited in your text

\pagenumbering{arabic}

\begin{spacing}{1.5} % Zeilenabstand: 1,5 fuer den Textteil

% Einleitung
\chapter{Einleitung}
\section{Einführung in das Thema}
Lorem ipsum dolor sit amet, consetetur sadipscing elitr, sed diam nonumy eirmod tempor invidunt ut labore et dolore magna aliquyam erat, sed diam voluptua. At vero eos et accusam et justo duo dolores et ea rebum. Stet clita kasd gubergren, no sea takimata sanctus est Lorem ipsum dolor sit amet. Lorem ipsum dolor sit amet, consetetur sadipscing elitr, sed diam nonumy eirmod tempor invidunt ut labore et dolore magna aliquyam erat, sed diam voluptua. At vero eos et accusam et justo duo dolores et ea rebum. Stet clita kasd gubergren, no sea takimata sanctus est Lorem ipsum dolor sit amet.

\section{Problemstellung und Ziel dieser Arbeit}

Lorem ipsum dolor sit amet, consetetur sadipscing elitr, sed diam nonumy eirmod tempor invidunt ut labore et dolore magna aliquyam erat, sed diam voluptua. At vero eos et accusam et justo duo dolores et ea rebum. Stet clita kasd gubergren, no sea takimata sanctus est Lorem ipsum dolor sit amet. Lorem ipsum dolor sit amet, consetetur sadipscing elitr, sed diam nonumy eirmod tempor invidunt ut labore et dolore magna aliquyam erat, sed diam voluptua. At vero eos et accusam et justo duo dolores et ea rebum. Stet clita kasd gubergren, no sea takimata sanctus est Lorem ipsum dolor sit amet.

\section{Aufbau der Arbeit}

\todo{An die fertige Arbeit anpassen} % chktex -10 this is an example. When writing your own TODOs, don't include this comment and you'll get a warning about every open TODO.

Lorem ipsum dolor sit amet, consetetur sadipscing elitr, sed diam nonumy eirmod tempor invidunt ut labore et dolore magna aliquyam erat, sed diam voluptua. At vero eos et accusam et justo duo dolores et ea rebum. Stet clita kasd gubergren, no sea takimata sanctus est Lorem ipsum dolor sit amet. Lorem ipsum dolor sit amet, consetetur sadipscing elitr, sed diam nonumy eirmod tempor invidunt ut labore et dolore magna aliquyam erat, sed diam voluptua.


%Grundlagen
\chapter{Grundlagen}
\section{Text}
Lorem ipsum dolor sit amet, consetetur sadipscing elitr, sed diam nonumy eirmod tempor invidunt ut labore et dolore magna aliquyam erat, sed diam voluptua.

\subsection{Fett}
\textbf{Lorem ipsum dolor sit amet, consetetur sadipscing elitr, sed diam nonumy eirmod tempor invidunt ut labore et dolore magna aliquyam erat, sed diam voluptua.}

\subsection{Kursiv}
\textit{Lorem ipsum dolor sit amet, consetetur sadipscing elitr, sed diam nonumy eirmod tempor invidunt ut labore et dolore magna aliquyam erat, sed diam voluptua.}

\subsection{Unterstrichen}
Lorem ipsum \underline{dolor} sit amet, consetetur sadipscing elitr, sed diam nonumy eirmod \underline{tempor} invidunt ut labore et dolore magna \underline{aliquyam} erat, sed diam voluptua.

\section{Fußnote}

Text mit Fußnote\footnote{Die Fußnote zum Text}

\section{Zitate}

Dies ist ein ganz kurzer Beispieltext\footnote{\cite{Richter2016}}. Und noch ein Zitat ohne Fußnote:~\cite{Jacobsen2017}
\\
Zitate auf Webseiten:~\cite{PlutoRed}
\\
Einträge ins Literaturverzeichnis (im File literatur.bib) können manuell geschrieben oder durch tools erzeugt werden, z.B. \url{http://www.literatur-generator.de}

\section{Aufzählung}

\begin{itemize}
\item\textit{Punkt 1:} Text
\item Punkt 2: \\ Text
\end{itemize}

\section{Glossar}
Hier werden \glspl{abk} aus dem Abkürzungsverzeichnis aufgerufen; bei der ersten Verwendung wird der volle Begriff ausgegeben und mit der Abkürzung in Verbindung gebracht, danach nur noch \emph{\gls{abk}} verwendet.

\gls{lorem} dolor sit amet, consetetur sadipscing elitr.


%Hauppteil
\chapter{Hauptteil}

\section{Tabelle}

\begin{center}
\tablehead{\textbf{Head1} & \textbf{Head2} & \textbf{Head3}\\}
\bottomcaption[Beschreibung]{Beschreibung. Quelle: Berger, Vorlesung, 2012, München }
\begin{supertabular}{c|c|c}
\hline
1 & 2 & 3\\
4 & 5 & 6\\
7 & 8 & 9\\
1 & 2 & 3\\
4 & 5 & 6\\
7 & 8 & 9\\
\end{supertabular}
\end{center}

\section{Bilder}

\begin{figure}[H]
\begin{center}
\includegraphics[scale=0.5]{resources/akad_bild1.jpg}
\caption[Akad]{Akad. Quelle: www.akad.de}
\end{center}
\end{figure}

%%Einkommentieren fuer Syntax Highlighting. In styles.tex muessen auch 2 Zeilen einkommentiert werden
%\subsection{Syntax Highlighting}
%\begin{figure}[h]
%\begin{minted}[linenos=true,bgcolor=bg]{php}
%<?php
%$title="Lorem";
%$desc = "Lorem Ipsum";
%include($_SERVER['DOCUMENT_ROOT'].'/header.php');
%?>
%\end{minted}
%\caption{Quellcode: Aufruf von header.php (PHP)}
%\label{abb:header}
%\end{figure}

\section{Formeln}

\begin{Formel}
\(a+b=c\)
\caption{AB Formel}
\end{Formel}


\section{Quellcode}
\begin{figure}[H]
\begin{lstlisting}[language=bash]
echo "Hello World"
\end{lstlisting}
\caption{Bash Hello World}
\end{figure}



%Schluss
\include{content/schluss}

\end{spacing}

\clearpage

\pagestyle{plain}


% Anhang 
\ifappendix
\appendix
\chapter{Anhang}

\section{Bilder}
\begin{figure}[H]
  \begin{center}
    \includegraphics[scale=0.5]{resources/akad_bild1.jpg}
    \caption[Akad Anhang]{Akad Anhang. Quelle: www.akad.de}
  \end{center}
\end{figure}

\section{Interview mit Prof.~Dr.~Google}
\begin{faq}
  \faqitem{Wie viele Einhörner passen auf ein Reiskorn?\label{faq:Reiskorn}}%
  {
    Meinten Sie ``Wie viele Reiskörner passen auf ein Reiskorn''?\par{}
    Ansonsten hätte ich ein hervorragendes Rezept für Milchreis mit Apfelmus für Sie\ldots
  }
  \faqitem{Rekursion\label{faq:ReiskornRekursion}}{Meinten Sie Rekursion\footnote{siehe auch \faqref{faq:Reiskorn}}?}
\end{faq}

%\section{Example Appendix}
%\label{app:example}
%\includepdf[pages=-]{resources/example.pdf}

\clearpage
\fi

% Literaturverzeichniss - Ab hier wieder Roemische Seitenzahlen

\pagenumbering{roman}
\setcounter{page}{\theromanPagenumber}
%\bibliographystyle{apalike}
%\bibliography{literatur}
\printbibliography[title=\prefbiblioname]
\onehalfspacing
\clearpage

\pagestyle{empty} 
\thispagestyle{empty}

\ifassignment
\else
   % this is not a simple assignment but a full thesis.
   \begin{center}
   {\Large Eidesstattliche Erkl"arung}
   \vspace*{4cm}\end{center}
   \noindent
   Ich versichere, dass ich die beiliegende Arbeit selbstst"andig verfasst, keine anderen als die angegebenen Quellen und Hilfsmittel benutzt sowie alle w"ortlich oder sinngem"a"s "ubernommenen Stellen in der Arbeit gekennzeichnet habe. 
   \vspace{3cm}

   \rule[0.5ex]{6.5cm}{1pt}\hfill\rule[0.5ex]{6.5cm}{1pt}
   \\
   \hspace*{0.8cm}(Datum, Ort)\hfill(Unterschrift)\hspace*{6.5cm-\widthof{(Unterschrift)}-0.8cm}

   \clearpage

   %Messbox zur Druckkontrolle:
   \newcommand{\Messbox}[2]{% Parameters: #1=Breite, #2=Hoehe
   \setlength{\unitlength}{1.0mm}%
   \begin{picture}(#1,#2)%
   \linethickness{0.05mm}%
   \put(0,0){\dashbox{0.2}(#1,#2)%
   {\parbox{#1mm}{%
   \centering\footnotesize 
   %{\bf MESSBOX}\\ 
   Breite $ = #1 {\ mm}$\\
   H\"ohe $ = #2 {\ mm}$
   }}}\end{picture}
   }

   \begin{center}
   \textbf{--- Druckgröße kontrollieren! ---}
   \\
   \Messbox{100}{50} % Angabe der Breite/Hoehe in mm
   \\
   \textbf{--- Diese Seite nach dem Druck entfernen! ---}
   \end{center}

\fi

\end{document}

