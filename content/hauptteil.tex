\chapter{Hauptteil}

\section{Tabelle}

\begin{center}
  \tablehead{\textbf{Head1} & \textbf{Head2} & \textbf{Head3}\\}
  \bottomcaption[Beschreibung]{Beschreibung. Quelle: Berger, Vorlesung, 2012, München }
  \begin{supertabular}{c|c|c}
    \hline  % chktex 44 suppress suggestion to use \toprule (TODO: determine how to handle this properly)
    1 & 2 & 3\\
    4 & 5 & 6\\
    7 & 8 & 9\\
    1 & 2 & 3\\
    4 & 5 & 6\\
    7 & 8 & 9\\
  \end{supertabular}
\end{center}

\section{Bilder}

\begin{figure}[H]
  \begin{center}
    \includegraphics[scale=0.5]{resources/akad_bild1.jpg}
    \caption[Akad]{Akad. Quelle: www.akad.de}
  \end{center}
\end{figure}

%%Einkommentieren fuer Syntax Highlighting. In styles.tex muessen auch 2 Zeilen einkommentiert werden
%\subsection{Syntax Highlighting}
%\begin{figure}[h]
%\begin{minted}[linenos=true,bgcolor=bg]{php}
%<?php
%$title="Lorem";
%$desc = "Lorem Ipsum";
%include($_SERVER['DOCUMENT_ROOT'].'/header.php');
%?>
%\end{minted}
%\caption{Quellcode: Aufruf von header.php (PHP)}
%\label{abb:header}
%\end{figure}

\section{Formeln}

\begin{Formel}
  \(a+b=c\)
  \caption{AB Formel}
\end{Formel}


\section{Quellcode}
\begin{figure}[H]
  \begin{lstlisting}[language=bash]
echo "Hello World"
\end{lstlisting}
  \caption{Bash Hello World}
\end{figure}

