% Hinweis: Wenn diese Arbeit in einem git repository geschrieben und auf github gesichert wird, besteht die Gefahr,
%          persönliche Daten (insbesondere Adresse und Matrikelnummer) unabsichtlich zu veröffentlichen.
% Um dies zu verhindern, sollten kritische Daten in einer oder mehreren separaten Dateien namens "secrets.tex" gesetzt werden (einfach von unten dorthin kopieren, einstellungen.tex muss nicht verändert werden).
% Diese können entweder im Projektverzeichnis selbst liegen, im Verzeichnis direkt darüber, oder im HOME-Verzeichnis ($HOME oder %USERPROFILE%).
% Es kann gemischt werden (z.B. Adresse in HOME, Studiengang im Projekt selbst), der jeweils zuerst gesetzte Wert hat Vorrang:
\IfFileExists{secrets.tex}{\input{secrets.tex}}{}                   % im Projekt-Ordner selbst (leicht zu packen/sichern/transferieren des Projekts, besonders für Dinge wie Titel und Thema des assignments)       % chktex 27 chktex tries to run this regardless of file existance
\IfFileExists{../secrets.tex}{\input{../secrets.tex}}{}             % direkt oberhalb des Projekt-Ordners (gültig für mehrere nebeneinander liegende Assignments, daher besonders für unveränderliche Daten wie Adresse)       % chktex 27 chktex tries to run this regardless of file existance
\IfFileExists{\string~/secrets.tex}{\input{\string~/secrets.tex}}{} % aus HOME Verzeichnis (gültig für alle Assignments auf dem Rechner egal wo sie liegen)       % chktex 27 chktex tries to run this regardless of file existance
% Im Folgenden nun die defaults für alle Werte, die nicht in einer der secrets.tex Dateien gesetzt wurden:


%%% Themenbezogene Daten (bei jedem neuen Assignment zu aktualisieren):

%Titel
\providecommand*{\Titel}{Thema des Assignments}

%PDF Beschreibung
\providecommand*{\pdfsubject}{Eine kurze Beschreibung, worum es geht}

%Betreff
\providecommand*{\Arbeitstyp}{Assignment im Modul ABC01}

%Betreuer
\providecommand*{\Betreuer}{Prof.~Dr.~Mustermann}

%Firma, in der die Arbeit absolviert wurde (für Sperrvermerk)
\providecommand*{\Firma}{Mustermann~AG}

%Bearbeitungszeit
\providecommand*{\Bearbeitungszeit}{8 Wochen}

%PDF Keywords
\providecommand*{\pdfkeywords}{akad, assignment, meta, information, pdf, hyperref, latex}



%%% personenbezogene Daten (bleiben gleich von Assignment zu Assignment, sollten vorzugsweise in secrets.tex gesetzt werden, siehe unten):

%Vor- und Nachname
\providecommand*{\Name}{Max Mustermann}

%Straße und Hausnummer
\providecommand*{\Strasse}{Musterstr.~1a}

%Plz und Ort
\providecommand*{\PlzOrt}{12345 Musterhausen}

%Email
\providecommand*{\Email}{max.mustermann@akad.de}

%Immatrikulationsnummer
\providecommand*{\Immatrikulationsnummer}{123456}

%Studiengang
\providecommand*{\Studiengang}{IT-Management}

%Akademischer Grad
\providecommand*{\Grad}{Master~of~Science~(M.~Sc.)}
%\providecommand*{\Grad}{Master~of~Engineering~(M.~Eng.)}
%\providecommand*{\Grad}{Bachelor~of~Science~(B.~Sc.)}
%\providecommand*{\Grad}{Bachelor~of~Engineering~(B.~Eng.)}



%%% generelle Einstellungen zur Darstellungsform
% Überschrift des Literaturverzeichnisses
\providecommand*{\prefbiblioname}{Literaturverzeichnis}

%% Nicht benötigte Zeilen mit % auskommentieren oder löschen:

%% darzustellende Verzeichnisse (können nicht in secrets.tex überschrieben werden):
\listoffigurestrue{} %% Abbildungsverzeichnis
\listoftablestrue{}  %% Tabellenverzeichnis
\acronymtrue{}       %% Abkürzungsverzeichnis
\glossarytrue{}      %% Glossar
% \listofformelntrue{} %% Formelverzeichnis
\appendixtrue{}      %% Anhang

%% Diese Vorlage wird für ein Assignment benutzt (statt für eine Abschlussarbeit)
\assignmenttrue{}

%% dieser Bericht beinhaltelt sensible Firmendaten und benötigt einen Sperrvermerk:
%\sperrvermerktrue
%% Sperrvermerke beziehen sich im Text direkt auf die Art des Berichts. Sollte ein Sperrvermerk notwendig sein, muss hier die richtige Wortwahl definiert werden:
%% Projektbericht:
%\providecommand*{\SperrvermerkBausteinEins}{Der nachfolgende Projektbericht}
%\providecommand*{\SperrvermerkBausteinZwei}{des Projektberichts}
%\providecommand*{\SperrvermerkBausteinDrei}{Der Projektbericht}
%% Diplom-/Master-/Bachelor-Arbeit:
%\providecommand*{\SperrvermerkGrad}{Diplom}
%\providecommand*{\SperrvermerkGrad}{Master}
%\providecommand*{\SperrvermerkGrad}{Bachelor}
%\providecommand*{\SperrvermerkBausteinEins}{Die nachfolgende \SperrvermerkGrad{}-Arbeit}
%\providecommand*{\SperrvermerkBausteinZwei}{der \SperrvermerkGrad{}-Arbeit}
%\providecommand*{\SperrvermerkBausteinDrei}{Die \SperrvermerkGrad{}-Arbeit}
