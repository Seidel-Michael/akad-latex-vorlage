% dummy comment for file-wide intellisense errors
%\usepackage{graphicx} % enables use of eps graphics (encapsulated PostScript). Activate if needed.
%\usepackage{newtx} % replacement of previously used "times" package (using Times font as default)
\usepackage{babel}
\usepackage{supertabular}
\usepackage{wrapfig}
\usepackage{multirow}
\usepackage[onehalfspacing]{setspace}
\usepackage{scrhack}  % fix float warning of KOMA produced when including listings
\usepackage{listings}
\usepackage{mathptmx}
\usepackage{geometry}
\usepackage{helvet}
\usepackage{courier}
\usepackage{setspace}
\usepackage{textcomp}
\usepackage[T1]{fontenc}
\usepackage[utf8]{inputenc}
\usepackage{float} % Notwendig fuer figure[h]
\usepackage[german=quotes]{csquotes}
\usepackage[style=alphabetic]{biblatex} % alternative for sort: iso-authoryear
\usepackage{pdfpages}
% Fuer Schriftart Arial
%\usepackage[scaled]{uarial}

% Installation der Arial Schriftart unter Linux.
% wget http://tug.org/fonts/getnonfreefonts/install-getnonfreefonts
% texlua install-getnonfreefonts
% getnonfreefonts -r
% getnonfreefonts arial-urw


%% Einstellungen fuer Quellcode Highlighting
\usepackage{color}
\definecolor{mygreen}{rgb}{0,0.6,0}
\definecolor{mygray}{rgb}{0.5,0.5,0.5}
\definecolor{mymauve}{rgb}{0.58,0,0.82}

\lstset{ %
  backgroundcolor=\color{white},   % choose the background color; you must add \usepackage{color} or \usepackage{xcolor}
  basicstyle=\footnotesize,        % the size of the fonts that are used for the code
  breakatwhitespace=false,         % sets if automatic breaks should only happen at whitespace
  breaklines=true,                 % sets automatic line breaking
  captionpos=b,                    % sets the caption-position to bottom
  commentstyle=\color{mygreen},    % comment style
  deletekeywords={...},            % if you want to delete keywords from the given language
  escapeinside={\%*}{*)},          % if you want to add LaTeX within your code
  extendedchars=true,              % lets you use non-ASCII characters; for 8-bits encodings only, does not work with UTF-8
 % frame=single,                    % adds a frame around the code
  keepspaces=true,                 % keeps spaces in text, useful for keeping indentation of code (possibly needs columns=flexible)
  keywordstyle=\color{blue},       % keyword style
  language=Octave,                 % the language of the code
  morekeywords={*,...},            % if you want to add more keywords to the set
  numbers=left,                    % where to put the line-numbers; possible values are (none, left, right)
  numbersep=5pt,                   % how far the line-numbers are from the code
  numberstyle=\tiny\color{mygray}, % the style that is used for the line-numbers
  rulecolor=\color{black},         % if not set, the frame-color may be changed on line-breaks within not-black text (e.g. comments (green here))
  showspaces=false,                % show spaces everywhere adding particular underscores; it overrides 'showstringspaces'
  showstringspaces=true,          % underline spaces within strings only
  showtabs=true,                  % show tabs within strings adding particular underscores
  stepnumber=1,                    % the step between two line-numbers. If it's 1, each line will be numbered
  stringstyle=\color{mymauve},     % string literal style
  tabsize=2,                       % sets default tabsize to 2 spaces
  title=\lstname,                   % show the filename of files included with \lstinputlisting; also try caption instead of title
  belowskip= 0pt 
}

% PDF Einstellungen für Verlinkungen

\usepackage[
	pdftitle={\Titel},
	pdfsubject={\pdfsubject},
	pdfauthor={\Name},
	pdfkeywords={\pdfkeywords}
	hyperfootnotes=false,
	colorlinks=true,
	linkcolor=black,
	urlcolor=black,
	citecolor=black
]{hyperref}

%%% Abkürzungsverzeichnis (Glossar) Neues Paket (kann nomencl und acronym ersetzen)
% muss nach hyperref eingebunden werden, um das Paket zu nutzen
% Abkürzungen werden nur im Glossar angezeigt, wenn sie im Dokument mindestens einmal genutzt wurden
\usepackage[
%	style=long,
	toc, % Glossar erscheint im Inhaltsverzeichnis
	acronym, % Setzt Akronyme in ein gesondertes Verzeichnis
%	footnote, % Setzt eine Fußnote beim ersten verwendet wird
%	nomain,
%	style=altlist,
	nopostdot, % löscht den schlusspunkt nach jeder description
]{glossaries}
\setglossarystyle{super}
\makeglossaries % Glossar generieren

\newfloat{Formel}{H}{for}

\renewcommand\UrlFont{\color{black}\rmfamily\itshape}

\renewcommand{\familydefault}{\rmdefault}
\newcommand{\bflabel}[1]{\normalfont{\normalsize{#1}}\hfill}