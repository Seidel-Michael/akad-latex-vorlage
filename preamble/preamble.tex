% dummy comment for file-wide intellisense errors
\usepackage{graphicx} % enables use of eps graphics (encapsulated PostScript). Activate if needed.
%\usepackage{newtx} % replacement of previously used "times" package (using Times font as default)
\usepackage{babel}
\usepackage{supertabular}
\usepackage{wrapfig}
\usepackage{multirow}
\usepackage[onehalfspacing]{setspace}
\usepackage{scrhack}  % fix float warning of KOMA produced when including listings
\usepackage{listings}
\usepackage{mathptmx}
\usepackage{geometry}
\usepackage{helvet}
\usepackage{courier}
\usepackage{setspace}
\usepackage{textcomp}
\usepackage{fontspec}
\usepackage[utf8]{inputenc}
\usepackage{float} % Notwendig fuer figure[h]
\usepackage{subcaption}
\usepackage[german=quotes]{csquotes}
\usepackage[style=authoryear-ibid,autocite=footnote]{biblatex} % alternative for sort: iso-authoryear
\usepackage{xurl} % better line-breaking than url package. Needs to be added after biblatex to work in bibliography.
\usepackage{pdfpages}
\usepackage{calc} % for calculations with text width
\usepackage{enumitem}
\usepackage{dblfnote}
\usepackage{tabularray}
\usepackage{rotating}
\usepackage{amssymb}
% Fuer Schriftart Arial
%\usepackage[scaled]{uarial}

% Installation der Arial Schriftart unter Linux.
% wget http://tug.org/fonts/getnonfreefonts/install-getnonfreefonts
% texlua install-getnonfreefonts
% getnonfreefonts -r
% getnonfreefonts arial-urw


%% Einstellungen fuer Quellcode Highlighting
\usepackage{color}
\definecolor{mygreen}{rgb}{0,0.6,0}
\definecolor{mygray}{rgb}{0.5,0.5,0.5}
\definecolor{mymauve}{rgb}{0.58,0,0.82}

\lstset{
	backgroundcolor=\color{white},   % choose the background color; you must add \usepackage{color} or \usepackage{xcolor}
	basicstyle=\footnotesize,        % the size of the fonts that are used for the code
	breakatwhitespace=false,         % sets if automatic breaks should only happen at whitespace
	breaklines=true,                 % sets automatic line breaking
	captionpos=b,                    % sets the caption-position to bottom
	commentstyle=\color{mygreen},    % comment style
	deletekeywords={...},            % if you want to delete keywords from the given language   % chktex 11 ignore suggestion to use \ldots
	escapeinside={\%*}{*)},          % if you want to add LaTeX within your code    % chktex 9 ignore unmatched parenthesis
	extendedchars=true,              % lets you use non-ASCII characters; for 8-bits encodings only, does not work with UTF-8
	% frame=single,                    % adds a frame around the code
	keepspaces=true,                 % keeps spaces in text, useful for keeping indentation of code (possibly needs columns=flexible)
	keywordstyle=\color{blue},       % keyword style
	language=Octave,                 % the language of the code
	morekeywords={*,...},            % if you want to add more keywords to the set   % chktex 11 ignore suggestion to use \ldots
	numbers=left,                    % where to put the line-numbers; possible values are (none, left, right)
	numbersep=5pt,                   % how far the line-numbers are from the code
	numberstyle=\tiny\color{mygray}, % the style that is used for the line-numbers
	rulecolor=\color{black},         % if not set, the frame-color may be changed on line-breaks within not-black text (e.g. comments (green here))
	showspaces=false,                % show spaces everywhere adding particular underscores; it overrides 'showstringspaces'
	showstringspaces=true,           % underline spaces within strings only
	showtabs=true,                   % show tabs within strings adding particular underscores
	stepnumber=1,                    % the step between two line-numbers. If it's 1, each line will be numbered
	stringstyle=\color{mymauve},     % string literal style
	tabsize=2,                       % sets default tabsize to 2 spaces
	title=\lstname,                  % show the filename of files included with \lstinputlisting; also try caption instead of title
	belowskip= 0pt,
}  % chktex 10 suppress false positive caused by a deliberately unmatched ')' above

% PDF Einstellungen für Verlinkungen

\usepackage[
	pdftitle={\Titel},
	pdfsubject={\pdfsubject},
	pdfauthor={\Name},
	pdfkeywords={\pdfkeywords}
	hyperfootnotes=false,
	colorlinks=true,
	linkcolor=black,
	urlcolor=black,
	citecolor=black
]{hyperref}

%%% Abkürzungsverzeichnis (Glossar) Neues Paket (kann nomencl und acronym ersetzen)
% muss nach hyperref eingebunden werden, um das Paket zu nutzen
% Abkürzungen werden nur im Glossar angezeigt, wenn sie im Dokument mindestens einmal genutzt wurden
\usepackage[
	% style=long,
	toc, % Glossar erscheint im Inhaltsverzeichnis
	acronym, % Setzt Akronyme in ein gesondertes Verzeichnis
	% footnote, % Setzt eine Fußnote beim ersten verwendet wird
	% nomain,
	% style=altlist,
	nopostdot, % löscht den schlusspunkt nach jeder description
]{glossaries}
\setglossarystyle{super}
\makeglossaries{} % Glossar generieren

\newfloat{Formel}{H}{for}

%% FAQ Umgebung, z.B. für Interview Protokolle im Appendix
\newenvironment{faq}{}{}
\DeclareSectionCommand[
	runin=false,                                        % start the answer in a new line
	afterskip=0.25\baselineskip plus -1ex minus -.2ex,  % chktex 1 commands cannot be terminated with curly braces in arguments
	beforeskip=-2.5ex plus -1ex minus -.2ex,
	indent=0pt,
	level=4,
	font=\usekomafont{paragraph}\itshape, %% using the same font as paragraph, but italic
	tocindent=10em,
	tocnumwidth=5em,
	counterwithin=subsubsection,
	style=section,
]{question}
\newcommand{\faqitem}[2]{\question{#1}{\setlength{\leftskip}{\parindent}#2\par}}
\def\questionautorefname{Frage} % for referring to labels within a question with \autoref, though I prefer referencing with the custom \faqref command
\newcommand*{\faqref}[1]{\hyperref[{#1}]{\appendixautorefname{}~\ref*{#1}\questionautorefname{}~``\nameref*{#1}''}}


\renewcommand\UrlFont{\color{black}\rmfamily\itshape} % chktex 6 ignore missing '\/' after \itshape

\addto\extrasngerman{%
	\def\subsectionautorefname{Abschnitt}%
	\def\subsubsectionautorefname{Abschnitt}%
}

\newcommand*{\fullref}[1]{\hyperref[{#1}]{\autoref*{#1} \nameref*{#1}}}

\renewcommand{\familydefault}{\rmdefault}
\newcommand{\bflabel}[1]{\normalfont{\normalsize{#1}}\hfill}




\usepackage{forloop}% used for \Qrating and \Qlines
\usepackage{ifthen}% used for \Qitem and \QItem
%%%%%%%%%%%%%%%%%%%%%%%%%%%%%%%%%%%%%%%%%%%%%%%%%%%%%%%%%%%%
%% Beginning of questionnaire command definitions %%
%%%%%%%%%%%%%%%%%%%%%%%%%%%%%%%%%%%%%%%%%%%%%%%%%%%%%%%%%%%%
%%
%% 2010, 2012 by Sven Hartenstein
%% mail@svenhartenstein.de
%% http://www.svenhartenstein.de
%%
%% Please be warned that this is NOT a full-featured framework for
%% creating (all sorts of) questionnaires. Rather, it is a small
%% collection of LaTeX commands that I found useful when creating a
%% questionnaire. Feel free to copy and adjust any parts you like.
%% Most probably, you will want to change the commands, so that they
%% fit your taste.
%%
%% Also note that I am not a LaTeX expert! Things can very likely be
%% done much more elegant than I was able to. If you have suggestions
%% about what can be improved please send me an email. I intend to
%% add good tipps to my website and to name contributers of course.
%%
%% 10/2012: Thanks to karathan for the suggestion to put \noindent
%% before \rule!

%% \Qq = Questionaire question. Oh, this is just too simple. It helps
%% making it easy to globally change the appearance of questions.
\newcommand{\Qq}[1]{\textbf{#1}}

%% \QO = Circle or box to be ticked. Used both by direct call and by
%% \Qrating and \Qlist.
\newcommand{\QO}{$\Box$}% or: $\ocircle$

%% \Qrating = Automatically create a rating scale with NUM steps, like
%% this: 0--0--0--0--0.
\newcounter{qr}
\newcommand{\Qrating}[1]{\QO\forloop{qr}{1}{\value{qr} < #1}{---\QO}}

%% \Qline = Again, this is very simple. It helps setting the line
%% thickness globally. Used both by direct call and by \Qlines.
\newcommand{\Qline}[1]{\noindent\rule{#1}{0.6pt}}

%% \Qlines = Insert NUM lines with width=\linewith. You can change the
%% \vskip value to adjust the spacing.
\newcounter{ql}
\newcommand{\Qlines}[1]{\forloop{ql}{0}{\value{ql}<#1}{\vskip0em\Qline{\linewidth}}}

%% \Qlist = This is an environment very similar to itemize but with
%% \QO in front of each list item. Useful for classical multiple
%% choice. Change leftmargin and topsep accourding to your taste.
\newenvironment{Qlist}{%
	\renewcommand{\labelitemi}{\QO}
	\begin{itemize}[leftmargin=1.5em,topsep=-.5em]
		}{%
	\end{itemize}
}

%% \Qtab = A "tabulator simulation". The first argument is the
%% distance from the left margin. The second argument is content which
%% is indented within the current row.
\newlength{\qt}
\newcommand{\Qtab}[2]{
	\setlength{\qt}{\linewidth}
	\addtolength{\qt}{-#1}
	\hfill\parbox[t]{\qt}{\raggedright #2}
}

%% \Qitem = Item with automatic numbering. The first optional argument
%% can be used to create sub-items like 2a, 2b, 2c, ... The item
%% number is increased if the first argument is omitted or equals 'a'.
%% You will have to adjust this if you prefer a different numbering
%% scheme. Adjust topsep and leftmargin as needed.
\newcounter{itemnummer}
\newcommand{\Qitem}[2][]{% #1 optional, #2 notwendig
	\ifthenelse{\equal{#1}{}}{\stepcounter{itemnummer}}{}
	\ifthenelse{\equal{#1}{a}}{\stepcounter{itemnummer}}{}
	\begin{enumerate}[topsep=2pt,leftmargin=2.8em]
		\item[\textbf{\arabic{itemnummer}#1.}] #2
	\end{enumerate}
}

%% \QItem = Like \Qitem but with alternating background color. This
%% might be error prone as I hard-coded some lengths (-5.25pt and
%% -3pt)! I do not yet understand why I need them.
\definecolor{bgodd}{rgb}{0.8,0.8,0.8}
\definecolor{bgeven}{rgb}{0.9,0.9,0.9}
\newcounter{itemoddeven}
\newlength{\gb}
\newcommand{\QItem}[2][]{% #1 optional, #2 notwendig
	\setlength{\gb}{\linewidth}
	\addtolength{\gb}{-5.25pt}
	\ifthenelse{\equal{\value{itemoddeven}}{0}}{%
		\noindent\colorbox{bgeven}{\hskip-3pt\begin{minipage}{\gb}\Qitem[#1]{#2}\end{minipage}}%
		\stepcounter{itemoddeven}%
	}{%
		\noindent\colorbox{bgodd}{\hskip-3pt\begin{minipage}{\gb}\Qitem[#1]{#2}\end{minipage}}%
		\setcounter{itemoddeven}{0}%
	}
}

%%%%%%%%%%%%%%%%%%%%%%%%%%%%%%%%%%%%%%%%%%%%%%%%%%%%%%%%%%%%
%% End of questionnaire command definitions %%
%%%%%%%%%%%%%%%%%%%%%%%%%%%%%%%%%%%%%%%%%%%%%%%%%%%%%%%%%%%%